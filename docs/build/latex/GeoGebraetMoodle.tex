%% Generated by Sphinx.
\def\sphinxdocclass{report}
\documentclass[letterpaper,10pt,french]{sphinxmanual}
\ifdefined\pdfpxdimen
   \let\sphinxpxdimen\pdfpxdimen\else\newdimen\sphinxpxdimen
\fi \sphinxpxdimen=.75bp\relax

\PassOptionsToPackage{warn}{textcomp}
\usepackage[utf8]{inputenc}
\ifdefined\DeclareUnicodeCharacter
 \ifdefined\DeclareUnicodeCharacterAsOptional
  \DeclareUnicodeCharacter{"00A0}{\nobreakspace}
  \DeclareUnicodeCharacter{"2500}{\sphinxunichar{2500}}
  \DeclareUnicodeCharacter{"2502}{\sphinxunichar{2502}}
  \DeclareUnicodeCharacter{"2514}{\sphinxunichar{2514}}
  \DeclareUnicodeCharacter{"251C}{\sphinxunichar{251C}}
  \DeclareUnicodeCharacter{"2572}{\textbackslash}
 \else
  \DeclareUnicodeCharacter{00A0}{\nobreakspace}
  \DeclareUnicodeCharacter{2500}{\sphinxunichar{2500}}
  \DeclareUnicodeCharacter{2502}{\sphinxunichar{2502}}
  \DeclareUnicodeCharacter{2514}{\sphinxunichar{2514}}
  \DeclareUnicodeCharacter{251C}{\sphinxunichar{251C}}
  \DeclareUnicodeCharacter{2572}{\textbackslash}
 \fi
\fi
\usepackage{cmap}
\usepackage[T1]{fontenc}
\usepackage{amsmath,amssymb,amstext}
\usepackage{babel}
\usepackage{times}
\usepackage[Sonny]{fncychap}
\ChNameVar{\Large\normalfont\sffamily}
\ChTitleVar{\Large\normalfont\sffamily}
\usepackage{sphinx}

\usepackage{geometry}

% Include hyperref last.
\usepackage{hyperref}
% Fix anchor placement for figures with captions.
\usepackage{hypcap}% it must be loaded after hyperref.
% Set up styles of URL: it should be placed after hyperref.
\urlstyle{same}
\addto\captionsfrench{\renewcommand{\contentsname}{Description du projet}}

\addto\captionsfrench{\renewcommand{\figurename}{Fig.}}
\addto\captionsfrench{\renewcommand{\tablename}{Tableau}}
\addto\captionsfrench{\renewcommand{\literalblockname}{Code source}}

\addto\captionsfrench{\renewcommand{\literalblockcontinuedname}{suite de la page précédente}}
\addto\captionsfrench{\renewcommand{\literalblockcontinuesname}{suite sur la page suivante}}

\addto\extrasfrench{\def\pageautorefname{page}}





\title{GeoGebra et Moodle IREM Marseille}
\date{mai 07, 2018}
\release{}
\author{Bruno Bourgine, Sylvain Ferrero, Pascal Padilla}
\newcommand{\sphinxlogo}{\vbox{}}
\renewcommand{\releasename}{}
\makeindex

\begin{document}

\maketitle
\sphinxtableofcontents
\phantomsection\label{\detokenize{index::doc}}


\sphinxstylestrong{Contenu du site}

Vous trouverez sur ce site des fichiers \sphinxstylestrong{GeoGebra} utilisés sur la plateforme \sphinxstylestrong{Moodle}.
La particularité de ces fichiers est qu’ils permettent d’évaluer l’élève automatiquement.
Chaque fichier comporte un système de score que Moodle peut récupérer. Pour cela, il faut
utiliser un plugin liant GeoGebra et Moodle.

\sphinxstylestrong{Qui sommes-nous ?}

Nous sommes des enseignants de maths/sciences regroupés au sein d’un groupe de recherche
de l”\sphinxhref{https://irem.univ-amu.fr/fr}{IREM de Marseille}.

Notre groupe, \sphinxstyleemphasis{Innovation, Expérimentation et Formation en Lycée Professionnel} (InEFLP)
a une partie de son travail consacrée aux modalités de cours innovantes. Nous explorons
actuellement la plateforme \sphinxstyleemphasis{Moodle}.

\sphinxhref{https://irem.univ-amu.fr/fr/groupes-travail/groupe-travail-innovation-experimentation-formation-lp}{Site du groupe InEFLP}.
\sphinxhref{https://irem.univ-amu.fr/fr}{\sphinxincludegraphics[width=150\sphinxpxdimen]{{logo-irem-2013-blanc}.jpg}}\sphinxhref{https://irem.univ-amu.fr/fr}{\sphinxincludegraphics[width=150\sphinxpxdimen]{{amu}.png}}

\bigskip\hrule\bigskip


\sphinxstylestrong{Table des matières du document}


\chapter{Description du projet}
\label{\detokenize{index:description-du-projet}}\label{\detokenize{index:bienvenus-sur-nos-pages-geogebra-et-moodle}}

\section{GeoGebra et Moodle ?}
\label{\detokenize{projet-description::doc}}\label{\detokenize{projet-description:geogebra-et-moodle}}
Il existe un plugin pour \sphinxstyleemphasis{Moodle} %
\begin{footnote}[1]\sphinxAtStartFootnote
Voir le site officiel du plugin : \sphinxurl{https://moodle.org/plugins/mod\_geogebra}
%
\end{footnote} permettant de créer des activités
\sphinxstyleemphasis{GeoGebra} et de sauvegarder son état. Il est alors possible :
\begin{itemize}
\item {} 
de sauvegarder la production d’un élève ;

\item {} 
d’évaluer manuellement ou \sphinxstylestrong{automatiquement} l’activité.

\end{itemize}


\subsection{GeoGebra}
\label{\detokenize{projet-description:geogebra}}
GeoGebra %
\begin{footnote}[2]\sphinxAtStartFootnote
Site GeoGebra : \sphinxurl{http://geogebra.org/}
%
\end{footnote} est un logiciel de géométrie dynamique permettant d’explorer,
d’expérimenter mais aussi de créer de la ressource pour les élèves.

Nous utilisons \sphinxstyleemphasis{GeoGebra} pour créer des exerciseurs. Généralement, nos activités
élèves sont construites de la façon suivante :
\begin{enumerate}
\item {} 
réaliser une tâche générée avec des \sphinxstylestrong{valeurs aléatoires}

\item {} 
saisir sa réponse et valider

\item {} 
si c’est incorrect, alors la réponse est affichée avec un corrigé

\item {} 
recommencer un certain nombre de fois les points 1 à 3

\item {} 
à la fin, obtenir un score sur son activité

\end{enumerate}

\noindent{\hspace*{\fill}\sphinxincludegraphics[width=0.500\linewidth]{{fig-proba-tirage}.png}\hspace*{\fill}}


\subsection{Moodle}
\label{\detokenize{projet-description:moodle}}
Moodle %
\begin{footnote}[3]\sphinxAtStartFootnote
Site francophone de Moodle : \sphinxurl{https://moodle.org/?lang=fr}
%
\end{footnote} est une application en ligne permettant de faire de la formation à
distance. De nombreux MOOC actuels l’utilisent. Cet outil est disponible dans de
nombreuses académie. Par exemple sur les académies de Nice et d’Aix-Marseille, Moodle
est intégré à l’ENE Atrium %
\begin{footnote}[4]\sphinxAtStartFootnote
Présentation de l’intégration de Moodle (et autres) avec
Atrium : \sphinxurl{https://www.atrium-paca.fr/web/assistance/acceder-a-moodle-chamilo-pronote-correlyce}
%
\end{footnote}.

Nous utilisons Moodle pour mettre les élèves en activités à l’aide d”\sphinxstylestrong{évaluations
formatives}. En effet, la notation choisie permet à l’éléve de s’entraîner, de se
former et d’être en réussite.

\begin{sphinxadmonition}{tip}{Astuce:}
Dans Moodle, nous utilisons la notation suivante :
\begin{itemize}
\item {} 
nombre maximum de tentatives : \sphinxstylestrong{illimité}

\item {} 
méthode d’évaluation : \sphinxstylestrong{Tentative la plus haute}

\end{itemize}

\noindent\sphinxincludegraphics[width=0.300\linewidth]{{fig-moodle}.png}
\end{sphinxadmonition}

Ainsi, l’élève fait autant de fois qu’il le désire l’actitivé proposée. Il n’est pas
obligé d’aller à son terme si c’est une activité répétitive. L’élève apprend de ses
erreurs car un corrigé l’accompagne à chaque tentative. Lorsqu’il le désire, l’élève
recommence l’activité et tente alors d’avoir un score maximal.

Cette façon d’évaluer est pour nous très pertinente.
L’élève est en activité, il est motivé car il sait qu’il peut réussir.
Il gagne en autonomie car, grâce au corrigé, il cherche à comprendre et à ne pas
reproduire ses erreurs.
En effet, chaque tentative est différente de la précédente car les fichiers sont
conçus à partir de \sphinxstylestrong{valeurs aléatoires}.


\subsection{Le plugin GeoGebra pour Moodle}
\label{\detokenize{projet-description:le-plugin-geogebra-pour-moodle}}
Ce plugin offre un nouveau type d’activité dans Moodle : \sphinxstyleemphasis{GeoGebra}.


\subsubsection{Utilisation de base}
\label{\detokenize{projet-description:utilisation-de-base}}
De base, il permet à l’enseignant de diffuser un fichier GeGebra (à envoyer dans
l’onglet \sphinxstyleemphasis{Contenu}).
L’élève entre alors dans l’activité et peut :
\begin{itemize}
\item {} 
modifier le fichier

\item {} 
sauvegarder ses modifications pour y revenir plus tard

\item {} 
envoyer sa production à l’enseignant.

\end{itemize}

L’enseigant pourra alors consulter les productions de chaque élève (un peu long car
à chaque fois le fichier doit s’ouvrir), ajouter commentaire et/ou note.


\subsubsection{Utilisation avancée}
\label{\detokenize{projet-description:utilisation-avancee}}
Une utilisation avancée du plugin permet la \sphinxstylestrong{notation automatique}.

\begin{sphinxadmonition}{tip}{Astuce:}
Pour activer la notation automatique, il faut :
\begin{itemize}
\item {} 
dans Moodle
* cocher \sphinxstyleemphasis{Activité auto-évaluée} dans l’onglet \sphinxstyleemphasis{Note}
* définir la note maximale

\item {} 
dans GeoGebra
* créer une variable \sphinxstyleemphasis{grade} qui aura une valeur entre 0 et la note maximale
* incrémenter la variable grade en fonction de l’activité de l’élève

\end{itemize}

\noindent\sphinxincludegraphics[width=0.300\linewidth]{{fig-moodle2}.png}

\noindent{\hspace*{\fill}\sphinxincludegraphics[width=0.600\linewidth]{{fig-moodle3}.png}}
\end{sphinxadmonition}


\bigskip\hrule\bigskip



\section{Ressources en lignes}
\label{\detokenize{projet-ressources:ressources-en-lignes}}\label{\detokenize{projet-ressources::doc}}
Nous remercions chaleureusement Joël Gauvain, fondateur du site \sphinxstylestrong{Mathématiques à Valin}, sans qui nous n’aurions pas sû utiliser cet outil formidable : GeoGebra + Moodle !

Voici un lien vers ses explications et ses ressources : énorme !

\sphinxhref{http://lycee-valin.fr/maths/exercices\_en\_ligne/moodle.html}{Mathématiques à Valin, Moodle et GeoGebra}


\section{À propos de cette documentation}
\label{\detokenize{projet-aPropos:a-propos-de-cette-documentation}}\label{\detokenize{projet-aPropos::doc}}
Nous publions cette documentation grâce aux outils suivant :
\begin{itemize}
\item {} 
\sphinxstylestrong{github} pour stocker nos données, travailler à plusieurs sur la rédaction,
relire et corriger.

\item {} 
\sphinxstylestrong{readthedoc} pour générer automatiquement cette documentation, héberger le site,
générer à la demande une version HTML, ePUB ou PDF.

\item {} 
\sphinxstylestrong{eclipse}  ou  \sphinxstylestrong{atom} pour travailler en local sur nos machines, et synchroniser à la demande.

\end{itemize}

Le format de documentation choisi pour rédiger cette doc est le \sphinxstylestrong{RestructuredText}.
C’est un format \sphinxstyleemphasis{relativement} simple dans sa syntaxe mais qui offre une structure
suffisament avancée pour permettre à des outils une génération automatique dans différents
formats.

Ainsi, à partir des mêmes fichiers sources, il est possible d’avoir un rendu de cette
documentation dans bien des formats.


\subsection{Environnement de travail}
\label{\detokenize{projet-aPropos:environnement-de-travail}}
Pour information, voici à quoi ressemble mon environnement de travail.
Je rédige cette page dans \sphinxstylestrong{Eclipse}. Avec une commande (CTRL+B), je génère un rendu
au format HTML que je visualise en direct dans la fenêtre de droite.

\noindent{\hspace*{\fill}\sphinxincludegraphics[width=0.650\linewidth]{{fig-aPropos}.png}\hspace*{\fill}}

C’est \sphinxstyleemphasis{plutôt} simple %
\begin{footnote}[1]\sphinxAtStartFootnote
Bon, j’avoue que j’ai passé pas mal de temps pour configurer tout ça.
%
\end{footnote} et fonctionnel.
\begin{enumerate}
\item {} 
Installer \sphinxstylestrong{Eclipse}

\item {} 
Dans le \sphinxstyleemphasis{Eclipse Marketplace}, installer \sphinxstylestrong{ReST Editor} %
\begin{footnote}[2]\sphinxAtStartFootnote
Pour info, c’est la version 1.0.5 chez moi
%
\end{footnote}

\item {} 
Synchroniser le dépot de \sphinxstylestrong{github} avec un dossier de travail dans votre répertoire
\sphinxstyleemphasis{Workplace}

\item {} 
Configurer \sphinxstyleemphasis{Eclipse} pour que le rendu se fasse par la commande \sphinxstyleemphasis{build all (CTRL + B)}

\end{enumerate}

\begin{sphinxadmonition}{note}{\label{projet-aPropos:index-0}À faire:}
Ajouter l’interface de rédaction de Bruno
\end{sphinxadmonition}


\bigskip\hrule\bigskip



\chapter{Probabilités et statistiques}
\label{\detokenize{index:probabilites-et-statistiques}}

\section{Médiane}
\label{\detokenize{proba stat - mediane:mediane}}\label{\detokenize{proba stat - mediane::doc}}
Voici deux exercices qui permettent de déterminer la médiane d’une série
de données brutes.
Le premier exercice donné ne comporte que des séries dont l’effectif total
est impair. Le second exercice propose aléatoirement des effectifs pairs
ou impairs.

\noindent{\hspace*{\fill}\sphinxincludegraphics[width=0.650\linewidth]{{fig-mediane}.png}\hspace*{\fill}}


\subsection{Fichiers à télécharger}
\label{\detokenize{proba stat - mediane:fichiers-a-telecharger}}

\begin{savenotes}\sphinxattablestart
\centering
\sphinxcapstartof{table}
\sphinxcaption{Médiane d’une série brute}\label{\detokenize{proba stat - mediane:id1}}
\sphinxaftercaption
\begin{tabular}[t]{|\X{30}{100}|\X{70}{100}|}
\hline
\sphinxstyletheadfamily 
Fichiers
&\sphinxstyletheadfamily 
Description
\\
\hline
\sphinxcode{\sphinxupquote{médiane impaire.ggb}}
&
déterminer la valeur médiane d’une série de 3, 5 ou 7 valeurs
\\
\hline
\sphinxcode{\sphinxupquote{mediane\_paire-impaire-10.ggb}}
&
déterminer la valeur médiane d’une série quelconque.
\sphinxstylestrong{Attention} il y a 10 questions.
\\
\hline
\end{tabular}
\par
\sphinxattableend\end{savenotes}


\subsection{Caractéristiques}
\label{\detokenize{proba stat - mediane:caracteristiques}}\begin{itemize}
\item {} 
exercices sur :
\begin{itemize}
\item {} 
5 points (5 questions) pour le premier

\item {} 
10 points (10 questions) pour le second

\end{itemize}

\item {} 
valeurs aléatoires (quantités, unités, questions, etc.)

\item {} 
notation automatique avec le plugin moodle : grâce à la variable \sphinxstyleemphasis{grade}

\end{itemize}


\section{Fréquences}
\label{\detokenize{proba-stat_fr_xe9quences:frequences}}\label{\detokenize{proba-stat_fr_xe9quences::doc}}
Série d’exercices sur le calcul de fréquence et l’étendue des fréquences pour une série d’échantillons.
\begin{enumerate}
\item {} 
Calcul de la fréquence d’un événement pour un échantillon de taille donnée (Pile ou Face).

\item {} 
Calcul de l’étendue des fréquences d’une série d’échantillon.

\item {} 
Calcul de l’étendue des fréquences d’une série d’échantillon (lecture graphique).

\end{enumerate}

\noindent{\hspace*{\fill}\sphinxincludegraphics[width=0.650\linewidth]{{fig-mediane}.png}\hspace*{\fill}}


\subsection{Fichiers à télécharger}
\label{\detokenize{proba-stat_fr_xe9quences:fichiers-a-telecharger}}

\begin{savenotes}\sphinxattablestart
\centering
\sphinxcapstartof{table}
\sphinxcaption{Fréquences}\label{\detokenize{proba-stat_fr_xe9quences:id1}}
\sphinxaftercaption
\begin{tabular}[t]{|\X{30}{100}|\X{70}{100}|}
\hline
\sphinxstyletheadfamily 
Fichiers
&\sphinxstyletheadfamily 
Description
\\
\hline
\sphinxcode{\sphinxupquote{médiane impaire.ggb}}
&
déterminer la fréquence de l’événement « côté face ».
\\
\hline
\sphinxcode{\sphinxupquote{mediane\_paire-impaire-10.ggb}}
&
déterminer l’étendue des fréquences.
\\
\hline
\sphinxcode{\sphinxupquote{mediane\_paire-impaire-10.ggb}}
&
déterminer l’étendue des fréquences (lecture graphique).
\\
\hline
\end{tabular}
\par
\sphinxattableend\end{savenotes}


\subsection{Caractéristiques}
\label{\detokenize{proba-stat_fr_xe9quences:caracteristiques}}\begin{itemize}
\item {} 
exercices sur :
\begin{itemize}
\item {} 
5 points (5 questions) pour le premier

\item {} 
valeurs aléatoires (quantités, unités, questions, etc.)

\item {} 
notation automatique avec le plugin moodle : grâce à la variable \sphinxstyleemphasis{grade}

\end{itemize}

\end{itemize}


\section{Tirages aléatoires}
\label{\detokenize{proba stat - tirage al_xe9a:tirages-aleatoires}}\label{\detokenize{proba stat - tirage al_xe9a::doc}}
Exercice de calcul de probabilité à partir d’un cas simple de tirage de boule dans une urne.

\noindent{\hspace*{\fill}\sphinxincludegraphics[width=0.650\linewidth]{{fig-proba-tirage}.png}\hspace*{\fill}}


\subsection{Fichiers à télécharger}
\label{\detokenize{proba stat - tirage al_xe9a:fichiers-a-telecharger}}

\begin{savenotes}\sphinxattablestart
\centering
\sphinxcapstartof{table}
\sphinxcaption{probabilité d’un événement}\label{\detokenize{proba stat - tirage al_xe9a:id1}}
\sphinxaftercaption
\begin{tabular}[t]{|\X{30}{100}|\X{70}{100}|}
\hline
\sphinxstyletheadfamily 
Fichiers
&\sphinxstyletheadfamily 
Description
\\
\hline
\sphinxcode{\sphinxupquote{proba tirage urne.ggb}}
&
calculer la probabilité d’un événement à partir des populations
\\
\hline
\end{tabular}
\par
\sphinxattableend\end{savenotes}


\subsection{Caractéristiques}
\label{\detokenize{proba stat - tirage al_xe9a:caracteristiques}}\begin{itemize}
\item {} 
exercices sur :
\begin{itemize}
\item {} 
5 points (5 questions) pour le premier

\item {} 
10 points (10 questions) pour le second

\end{itemize}

\item {} 
valeurs aléatoires (quantités, unités, questions, etc.)

\item {} 
notation automatique avec le plugin moodle : grâce à la variable \sphinxstyleemphasis{grade}

\end{itemize}


\chapter{Analyse et Algèbre}
\label{\detokenize{index:analyse-et-algebre}}

\section{Proportionnalité}
\label{\detokenize{analyse alg_xe8bre - proportionnalit_xe9:proportionnalite}}\label{\detokenize{analyse alg_xe8bre - proportionnalit_xe9::doc}}
Pour commencer, un exercice de calcul de la quatrième proportionnelle.

\noindent{\hspace*{\fill}\sphinxincludegraphics[width=0.650\linewidth]{{fig-quatrieme_prop}.png}\hspace*{\fill}}


\subsection{Fichiers à télécharger}
\label{\detokenize{analyse alg_xe8bre - proportionnalit_xe9:fichiers-a-telecharger}}

\begin{savenotes}\sphinxattablestart
\centering
\sphinxcapstartof{table}
\sphinxcaption{Calcul de proportionnalité}\label{\detokenize{analyse alg_xe8bre - proportionnalit_xe9:id1}}
\sphinxaftercaption
\begin{tabulary}{\linewidth}[t]{|T|T|}
\hline
\sphinxstyletheadfamily 
Fichier
&\sphinxstyletheadfamily 
Description
\\
\hline
\sphinxcode{\sphinxupquote{quatrieme prop.ggb}}
&
pourcentage direct (calculer une quantité à partir du taux)
\\
\hline
\end{tabulary}
\par
\sphinxattableend\end{savenotes}


\subsection{Caractéristiques}
\label{\detokenize{analyse alg_xe8bre - proportionnalit_xe9:caracteristiques}}\begin{itemize}
\item {} 
exercices sur 5 points (5 questions)

\item {} 
valeurs aléatoires (quantités, unités, questions, etc.)

\item {} 
notation automatique avec le plugin moodle : grâce à la variable \sphinxstyleemphasis{grade}

\end{itemize}


\section{Pourcentages}
\label{\detokenize{analyse alg_xe8bre - pourcentages:pourcentages}}\label{\detokenize{analyse alg_xe8bre - pourcentages::doc}}
Nous vous proposons 7 exercices sur les pourcentages.
Voici par exemple l’exercice 7 qui nous a servi de \sphinxstylestrong{synthèse}.

\begin{sphinxadmonition}{warning}{Avertissement:}
Attention, le bloc ci-dessous est dynamique et peut mal s’afficher…
\end{sphinxadmonition}




\subsection{Fichiers à télécharger}
\label{\detokenize{analyse alg_xe8bre - pourcentages:fichiers-a-telecharger}}

\begin{savenotes}\sphinxattablestart
\centering
\sphinxcapstartof{table}
\sphinxcaption{Pourcentages directs et indirects}\label{\detokenize{analyse alg_xe8bre - pourcentages:id1}}
\sphinxaftercaption
\begin{tabulary}{\linewidth}[t]{|T|T|}
\hline
\sphinxstyletheadfamily 
Fichier
&\sphinxstyletheadfamily 
Description
\\
\hline
\sphinxcode{\sphinxupquote{pourcentage1.ggb}}
&
pourcentage direct (calculer une quantité à partir du taux)
\\
\hline
\sphinxcode{\sphinxupquote{pourcentage2.ggb}}
&
calculer un taux (à partir des quantités initiales et finales)
\\
\hline
\sphinxcode{\sphinxupquote{pourcentage3.ggb}}
&
pourcentage indirect (calculer la quantité initiale à partir du taux)
\\
\hline
\end{tabulary}
\par
\sphinxattableend\end{savenotes}


\begin{savenotes}\sphinxattablestart
\centering
\sphinxcapstartof{table}
\sphinxcaption{Augmentations, diminutions de pourcentages}\label{\detokenize{analyse alg_xe8bre - pourcentages:id2}}
\sphinxaftercaption
\begin{tabular}[t]{|\X{1}{3}|\X{2}{3}|}
\hline

\sphinxcode{\sphinxupquote{pourcentage4.ggb}}
&
calculer une quantité après une augmentation/réduction
\\
\hline
\sphinxcode{\sphinxupquote{pourcentage5.ggb}}
&
calculer un taux d’augmentation/réduction
\\
\hline
\sphinxcode{\sphinxupquote{pourcentage6.ggb}}
&
calculer la quantité initiale connaissant la quantité finale et le taux d’augmentation/réduction
\\
\hline
\end{tabular}
\par
\sphinxattableend\end{savenotes}


\begin{savenotes}\sphinxattablestart
\centering
\sphinxcapstartof{table}
\sphinxcaption{Synthèse}\label{\detokenize{analyse alg_xe8bre - pourcentages:id3}}
\sphinxaftercaption
\begin{tabular}[t]{|\X{1}{3}|\X{2}{3}|}
\hline

\sphinxcode{\sphinxupquote{pourcentage7.ggb}}
&
la synthèse : un mélange de tous les cas précédents
\\
\hline
\end{tabular}
\par
\sphinxattableend\end{savenotes}


\subsection{Caractéristiques}
\label{\detokenize{analyse alg_xe8bre - pourcentages:caracteristiques}}\begin{itemize}
\item {} 
exercices sur 5 points (5 questions)

\item {} 
valeurs aléatoires (quantités, unités, questions, etc.)

\item {} 
notation automatique avec le plugin moodle : grâce à la variable \sphinxstyleemphasis{grade}

\end{itemize}


\section{Équation du premier degré}
\label{\detokenize{analyse alg_xe8bre - _xe9quation degr_xe91:equation-du-premier-degre}}\label{\detokenize{analyse alg_xe8bre - _xe9quation degr_xe91::doc}}
Exercice sur la résolution d’équations du premier degré à une inconnue.
Aléatoirement sont proposées des équations de type ax+b=0 ou ax+b=cx+d.

\noindent{\hspace*{\fill}\sphinxincludegraphics[width=0.650\linewidth]{{fig-equa1}.png}\hspace*{\fill}}


\subsection{Fichiers à télécharger}
\label{\detokenize{analyse alg_xe8bre - _xe9quation degr_xe91:fichiers-a-telecharger}}

\begin{savenotes}\sphinxattablestart
\centering
\sphinxcapstartof{table}
\sphinxcaption{Calcul d’images}\label{\detokenize{analyse alg_xe8bre - _xe9quation degr_xe91:id1}}
\sphinxaftercaption
\begin{tabular}[t]{|\X{30}{100}|\X{70}{100}|}
\hline
\sphinxstyletheadfamily 
Fichiers
&\sphinxstyletheadfamily 
Description
\\
\hline
\sphinxcode{\sphinxupquote{resolution equation degre1.ggb}}
&
calculer des équations du premier degré à une inconnue
\\
\hline
\end{tabular}
\par
\sphinxattableend\end{savenotes}


\subsection{Caractéristiques}
\label{\detokenize{analyse alg_xe8bre - _xe9quation degr_xe91:caracteristiques}}\begin{itemize}
\item {} 
exercices sur 5 points (5 questions)

\item {} 
valeurs aléatoires (quantités, unités, questions, etc.)

\item {} 
notation automatique avec le plugin moodle : grâce à la variable \sphinxstyleemphasis{grade}

\end{itemize}


\section{Fonctions}
\label{\detokenize{analyse alg_xe8bre - fonctions:fonctions}}\label{\detokenize{analyse alg_xe8bre - fonctions::doc}}

\subsection{Fichiers à télécharger}
\label{\detokenize{analyse alg_xe8bre - fonctions:fichiers-a-telecharger}}

\begin{savenotes}\sphinxattablestart
\centering
\sphinxcapstartof{table}
\sphinxcaption{Calcul d’images}\label{\detokenize{analyse alg_xe8bre - fonctions:id1}}
\sphinxaftercaption
\begin{tabular}[t]{|\X{30}{100}|\X{70}{100}|}
\hline
\sphinxstyletheadfamily 
Fichiers
&\sphinxstyletheadfamily 
Description
\\
\hline
\sphinxcode{\sphinxupquote{calcul-image-affine.ggb}}
&
calculer l’image à partir de l’expression algébrique d’une fonction affine
\\
\hline
\sphinxcode{\sphinxupquote{calcul-image-polynôme.ggb}}
&
calculer l’image à partir de l’éxpression algébrique d’une fonction rationnelle
\\
\hline
\end{tabular}
\par
\sphinxattableend\end{savenotes}


\begin{savenotes}\sphinxattablestart
\centering
\sphinxcapstartof{table}
\sphinxcaption{Sens de variation}\label{\detokenize{analyse alg_xe8bre - fonctions:id2}}
\sphinxaftercaption
\begin{tabular}[t]{|\X{30}{100}|\X{70}{100}|}
\hline
\sphinxstyletheadfamily 
Fichiers
&\sphinxstyletheadfamily 
Description
\\
\hline
\sphinxcode{\sphinxupquote{sens de variation fonction affine graphique.ggb}}
&
déterminer le sens de variation d’une fonction affine à partir de sa représentation graphique
\\
\hline
\sphinxcode{\sphinxupquote{sens de variation fonction affine.ggb}}
&
déterminer le sens de variation d’une fonction affine à partir de l’expression algébrique
\\
\hline
\end{tabular}
\par
\sphinxattableend\end{savenotes}


\subsection{Caractéristiques}
\label{\detokenize{analyse alg_xe8bre - fonctions:caracteristiques}}\begin{itemize}
\item {} 
exercices sur 5 points (5 questions)

\item {} 
valeurs aléatoires (quantités, unités, questions, etc.)

\item {} 
notation automatique avec le plugin moodle : grâce à la variable \sphinxstyleemphasis{grade}

\end{itemize}


\chapter{Géométrie}
\label{\detokenize{index:geometrie}}

\section{Pythagore}
\label{\detokenize{geom-pythagore:pythagore}}\label{\detokenize{geom-pythagore::doc}}
Calculs de longueurs dans le triangle rectangle avec le théorème de Pythagore, le calcul
attendu concerne aléatoirement l’hypothénuse ou un côté de l’angle droit.

\noindent{\hspace*{\fill}\sphinxincludegraphics[width=0.650\linewidth]{{fig-pythagore_longueurs}.png}\hspace*{\fill}}


\subsection{Fichiers à télécharger}
\label{\detokenize{geom-pythagore:fichiers-a-telecharger}}

\begin{savenotes}\sphinxattablestart
\centering
\sphinxcapstartof{table}
\sphinxcaption{Calcul de longueurs dans le triangle rectangle avec le théorème de Pythagore}\label{\detokenize{geom-pythagore:id1}}
\sphinxaftercaption
\begin{tabular}[t]{|\X{2}{4}|\X{2}{4}|}
\hline
\sphinxstyletheadfamily 
Fichier
&\sphinxstyletheadfamily 
Description
\\
\hline
\sphinxcode{\sphinxupquote{pythagore longueurs.ggb}}
&
calculer des longueurs avec le théorème de Pythagore
\\
\hline
\end{tabular}
\par
\sphinxattableend\end{savenotes}


\subsection{Caractéristiques}
\label{\detokenize{geom-pythagore:caracteristiques}}\begin{itemize}
\item {} 
exercices sur 5 points (5 questions)

\item {} 
valeurs aléatoires (quantités, unités, questions, etc.)

\item {} 
notation automatique avec le plugin moodle : grâce à la variable \sphinxstyleemphasis{grade}

\item {} 
correction personnalisée

\end{itemize}


\section{Thalès}
\label{\detokenize{geom-thales:thales}}\label{\detokenize{geom-thales::doc}}
Calculs de longueurs dans des triangles semblables avec le théorème de Thalès.
Différentes configurations sont proposées de façon aléatoires.

\noindent{\hspace*{\fill}\sphinxincludegraphics[width=0.650\linewidth]{{fig-Thales}.png}\hspace*{\fill}}


\subsection{Fichiers à télécharger}
\label{\detokenize{geom-thales:fichiers-a-telecharger}}

\begin{savenotes}\sphinxattablestart
\centering
\sphinxcapstartof{table}
\sphinxcaption{Calcul de longueurs de triangles  avec le théorème de Thalès}\label{\detokenize{geom-thales:id1}}
\sphinxaftercaption
\begin{tabular}[t]{|\X{2}{4}|\X{2}{4}|}
\hline
\sphinxstyletheadfamily 
Fichier
&\sphinxstyletheadfamily 
Description
\\
\hline
\sphinxcode{\sphinxupquote{thales longueurs.ggb}}
&
calculer des longueurs avec le théorème de Thalès
\\
\hline
\end{tabular}
\par
\sphinxattableend\end{savenotes}


\subsection{Caractéristiques}
\label{\detokenize{geom-thales:caracteristiques}}\begin{itemize}
\item {} 
exercices sur 5 points (5 questions)

\item {} 
valeurs aléatoires (quantités, unités, questions, etc.)

\item {} 
notation automatique avec le plugin moodle : grâce à la variable \sphinxstyleemphasis{grade}

\item {} 
correction personnalisée

\end{itemize}


\section{Vecteurs}
\label{\detokenize{geom-vecteur::doc}}\label{\detokenize{geom-vecteur:vecteurs}}
Ici, deux séries d’exercices :
\begin{itemize}
\item {} 
{\hyperref[\detokenize{geom-vecteur:plan}]{\sphinxcrossref{\DUrole{std,std-ref}{Vecteurs dans le plan}}}}

\item {} 
{\hyperref[\detokenize{geom-vecteur:espace}]{\sphinxcrossref{\DUrole{std,std-ref}{Vecteurs dans l’espace}}}}

\end{itemize}


\subsection{Vecteurs dans le plan}
\label{\detokenize{geom-vecteur:plan}}\label{\detokenize{geom-vecteur:vecteurs-dans-le-plan}}
Série d’exercices sur les vecteurs dans le plan.
\begin{enumerate}
\item {} 
Calcul des coordonnées d’un vecteur plan à partir des coordonnées de 2 points.

\item {} 
Calcul de la norme d’un vecteur plan à partir des coordonnées de celui-ci.

\end{enumerate}

\noindent{\hspace*{\fill}\sphinxincludegraphics[width=0.650\linewidth]{{fig-vecteur}.png}\hspace*{\fill}}


\subsubsection{Fichiers à télécharger}
\label{\detokenize{geom-vecteur:fichiers-a-telecharger}}

\begin{savenotes}\sphinxattablestart
\centering
\sphinxcapstartof{table}
\sphinxcaption{Coordonnées et normes dans le plan}\label{\detokenize{geom-vecteur:id2}}
\sphinxaftercaption
\begin{tabular}[t]{|\X{2}{4}|\X{2}{4}|}
\hline
\sphinxstyletheadfamily 
Fichier
&\sphinxstyletheadfamily 
Description
\\
\hline
\sphinxcode{\sphinxupquote{coordonnées vecteur plan.ggb}}
&
déterminer les coordonnées d’un vecteur plan
\\
\hline
\sphinxcode{\sphinxupquote{norme vecteur plan.ggb}}
&
calculer la norme d’un vecteur dans le plan (à partir des coordonnées)
\\
\hline
\end{tabular}
\par
\sphinxattableend\end{savenotes}


\subsection{Vecteurs dans l’espace}
\label{\detokenize{geom-vecteur:vecteurs-dans-l-espace}}\label{\detokenize{geom-vecteur:espace}}
Série d’exercices sur les vecteurs dans l’espace : calcul de coordonnées et de normes.

\noindent{\hspace*{\fill}\sphinxincludegraphics[width=0.650\linewidth]{{fig-vecteur3D}.png}\hspace*{\fill}}


\subsubsection{Fichiers à télécharger}
\label{\detokenize{geom-vecteur:id1}}

\begin{savenotes}\sphinxattablestart
\centering
\sphinxcapstartof{table}
\sphinxcaption{Coordonnées et normes dans l’espace”}\label{\detokenize{geom-vecteur:id3}}
\sphinxaftercaption
\begin{tabular}[t]{|\X{2}{4}|\X{2}{4}|}
\hline
\sphinxstyletheadfamily 
Fichier
&\sphinxstyletheadfamily 
Description
\\
\hline
\sphinxcode{\sphinxupquote{coordonnées vecteur espace.ggb}}
&
déterminer les coordonnées d’un vecteur dans l’espace
\\
\hline
\sphinxcode{\sphinxupquote{norme vecteur espace.ggb}}
&
calculer la norme d’un vecteur dans le plan (à partir de ses coordonnées)
\\
\hline
\sphinxcode{\sphinxupquote{norme2 vecteur espace.ggb}}
&
calculer la norme d’un vecteur dans le plan (à partir des coordonnées de ses extrémités)
\\
\hline
\end{tabular}
\par
\sphinxattableend\end{savenotes}


\subsection{Caractéristiques}
\label{\detokenize{geom-vecteur:caracteristiques}}\begin{itemize}
\item {} 
exercices sur 5 points (5 questions)

\item {} 
valeurs aléatoires (quantités, unités, questions, etc.)

\item {} 
notation automatique avec le plugin moodle : grâce à la variable \sphinxstyleemphasis{grade}

\item {} 
correction personnalisée

\end{itemize}


\chapter{Sciences}
\label{\detokenize{index:sciences}}

\section{Mécanique}
\label{\detokenize{sciences-mecanique:mecanique}}\label{\detokenize{sciences-mecanique::doc}}
Une série d’exercice sur le calcul du poids, de la masse ou de la constante
gravitationnelle à partir de la relation :
\begin{equation*}
\begin{split}p = m \times g\end{split}
\end{equation*}
\noindent{\hspace*{\fill}\sphinxincludegraphics[width=0.650\linewidth]{{fig-p=mg}.png}\hspace*{\fill}}


\subsection{Fichiers à télécharger}
\label{\detokenize{sciences-mecanique:fichiers-a-telecharger}}

\begin{savenotes}\sphinxattablestart
\centering
\sphinxcapstartof{table}
\sphinxcaption{Poids, masse et constante gravitationnelle}\label{\detokenize{sciences-mecanique:id1}}
\sphinxaftercaption
\begin{tabular}[t]{|\X{30}{100}|\X{70}{100}|}
\hline
\sphinxstyletheadfamily 
Fichiers
&\sphinxstyletheadfamily 
Description
\\
\hline
\sphinxcode{\sphinxupquote{p=mg.ggb}}
&
déterminer P, m ou g (niveau 1)
\\
\hline
\sphinxcode{\sphinxupquote{p=mg niveau 2.ggb}}
&
déterminer P, m ou g avec conversion simple d’unité (niveau 2)
\sphinxstylestrong{Attention} il y a 10 questions.
\\
\hline
\end{tabular}
\par
\sphinxattableend\end{savenotes}


\subsection{Caractéristiques}
\label{\detokenize{sciences-mecanique:caracteristiques}}\begin{itemize}
\item {} 
exercices sur 5 points (5 questions)

\item {} 
valeurs aléatoires (quantités, unités, questions, etc.)

\item {} 
notation automatique avec le plugin moodle : grâce à la variable \sphinxstyleemphasis{grade}

\end{itemize}


\section{Énergétique}
\label{\detokenize{sciences-energie:energetique}}\label{\detokenize{sciences-energie::doc}}
Série d’exercices sur les échanges énergétiques lors de changements de températures
\begin{enumerate}
\item {} 
chaleur latente

\item {} 
capacité thermique et changement d’état

\item {} 
synthèse

\end{enumerate}


\begin{savenotes}\sphinxattablestart
\centering
\sphinxcapstartof{table}
\sphinxcaption{Captures d’écran des 3 fichiers}\label{\detokenize{sciences-energie:id1}}
\sphinxaftercaption
\begin{tabulary}{\linewidth}[t]{|T|T|T|}
\hline

\noindent\sphinxincludegraphics[width=1.000\linewidth]{{fig-chgt01}.png}
&
\noindent\sphinxincludegraphics[width=1.000\linewidth]{{fig-chgt02}.png}
&
\noindent\sphinxincludegraphics[width=1.000\linewidth]{{fig-chgt03}.png}
\\
\hline
\end{tabulary}
\par
\sphinxattableend\end{savenotes}


\subsection{Fichiers à télécharger}
\label{\detokenize{sciences-energie:fichiers-a-telecharger}}

\begin{savenotes}\sphinxattablestart
\centering
\sphinxcapstartof{table}
\sphinxcaption{Poids, masse et constante gravitationnelle}\label{\detokenize{sciences-energie:id2}}
\sphinxaftercaption
\begin{tabular}[t]{|\X{30}{100}|\X{70}{100}|}
\hline
\sphinxstyletheadfamily 
Fichiers
&\sphinxstyletheadfamily 
Description
\\
\hline
\sphinxcode{\sphinxupquote{capacité thermique.ggb}}
&
utiliser la capacité thermique pour déterminer la quantité d’énergie échangée
\\
\hline
\sphinxcode{\sphinxupquote{chaleur latente.ggb}}
&
utiliser la chaleur latente et le changement d’état à venir
pour déterminer la quantité d’énergie échangée
\\
\hline
\sphinxcode{\sphinxupquote{synthèse.ggb}}
&
\sphinxstylestrong{synthèse} : utiliser changement d’été, chaleur latente et capacité thermique
pour déterminer la quantité d’énergie échangée
\\
\hline
\end{tabular}
\par
\sphinxattableend\end{savenotes}


\subsection{Caractéristiques}
\label{\detokenize{sciences-energie:caracteristiques}}\begin{itemize}
\item {} 
exercices sur 5 points (5 questions)

\item {} 
valeurs aléatoires (quantités, unités, questions, etc.)

\item {} 
notation automatique avec le plugin moodle : grâce à la variable \sphinxstyleemphasis{grade}

\end{itemize}


\bigskip\hrule\bigskip

\begin{itemize}
\item {} 
\DUrole{xref,std,std-ref}{genindex}

\item {} 
\DUrole{xref,std,std-ref}{search}

\end{itemize}



\renewcommand{\indexname}{Index}
\printindex
\end{document}