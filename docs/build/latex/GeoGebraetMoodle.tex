%% Generated by Sphinx.
\def\sphinxdocclass{report}
\documentclass[letterpaper,10pt,french]{sphinxmanual}
\ifdefined\pdfpxdimen
   \let\sphinxpxdimen\pdfpxdimen\else\newdimen\sphinxpxdimen
\fi \sphinxpxdimen=.75bp\relax

\PassOptionsToPackage{warn}{textcomp}
\usepackage[utf8]{inputenc}
\ifdefined\DeclareUnicodeCharacter
 \ifdefined\DeclareUnicodeCharacterAsOptional
  \DeclareUnicodeCharacter{"00A0}{\nobreakspace}
  \DeclareUnicodeCharacter{"2500}{\sphinxunichar{2500}}
  \DeclareUnicodeCharacter{"2502}{\sphinxunichar{2502}}
  \DeclareUnicodeCharacter{"2514}{\sphinxunichar{2514}}
  \DeclareUnicodeCharacter{"251C}{\sphinxunichar{251C}}
  \DeclareUnicodeCharacter{"2572}{\textbackslash}
 \else
  \DeclareUnicodeCharacter{00A0}{\nobreakspace}
  \DeclareUnicodeCharacter{2500}{\sphinxunichar{2500}}
  \DeclareUnicodeCharacter{2502}{\sphinxunichar{2502}}
  \DeclareUnicodeCharacter{2514}{\sphinxunichar{2514}}
  \DeclareUnicodeCharacter{251C}{\sphinxunichar{251C}}
  \DeclareUnicodeCharacter{2572}{\textbackslash}
 \fi
\fi
\usepackage{cmap}
\usepackage[T1]{fontenc}
\usepackage{amsmath,amssymb,amstext}
\usepackage{babel}
\usepackage{times}
\usepackage[Sonny]{fncychap}
\ChNameVar{\Large\normalfont\sffamily}
\ChTitleVar{\Large\normalfont\sffamily}
\usepackage{sphinx}

\usepackage{geometry}

% Include hyperref last.
\usepackage{hyperref}
% Fix anchor placement for figures with captions.
\usepackage{hypcap}% it must be loaded after hyperref.
% Set up styles of URL: it should be placed after hyperref.
\urlstyle{same}
\addto\captionsfrench{\renewcommand{\contentsname}{Description du projet}}

\addto\captionsfrench{\renewcommand{\figurename}{Fig.}}
\addto\captionsfrench{\renewcommand{\tablename}{Tableau}}
\addto\captionsfrench{\renewcommand{\literalblockname}{Code source}}

\addto\captionsfrench{\renewcommand{\literalblockcontinuedname}{suite de la page précédente}}
\addto\captionsfrench{\renewcommand{\literalblockcontinuesname}{suite sur la page suivante}}

\addto\extrasfrench{\def\pageautorefname{page}}

\setcounter{tocdepth}{0}



\title{GeoGebra et Moodle Documentation}
\date{avr. 28, 2018}
\release{}
\author{Bruno Bourgine, Sylvain Ferrero, Pascal Padilla}
\newcommand{\sphinxlogo}{\vbox{}}
\renewcommand{\releasename}{}
\makeindex

\begin{document}

\maketitle
\sphinxtableofcontents
\phantomsection\label{\detokenize{index::doc}}



\chapter{Des ressources Moodle / GeoGebra}
\label{\detokenize{index:bienvenus-sur-nos-pages-geogebra-et-moodle}}\label{\detokenize{index:des-ressources-moodle-geogebra}}

\section{Contenu du site}
\label{\detokenize{index:contenu-du-site}}
Vous trouverez sur ce site des fichiers \sphinxstylestrong{GeoGebra} utilisés sur la plateforme \sphinxstylestrong{Moodle}.
La particularité de ces fichiers est qu’ils permettent d’évaluer l’élève automatiquement. Chaque fichier comporte un système de score que Moodle peut récupérer. Pour cela, il faut utiliser un plugin liant GeoGebra et Moodle


\section{Qui sommes-nous ?}
\label{\detokenize{index:qui-sommes-nous}}
Nous sommes des enseignants de maths/sciences regroupés au sein d’un groupe de recherche de l”\sphinxhref{https://irem.univ-amu.fr/fr}{IREM de Marseille}.

Notre groupe, \sphinxstyleemphasis{Innovation, Expérimentation et Formation en Lycée Professionnel} (InEFLP) a une partie de son travail consacré aux modalités de cours innovantes. Nous explorons actuellement la plateforme \sphinxstyleemphasis{Moodle}.

\sphinxhref{https://irem.univ-amu.fr/fr/groupes-travail/groupe-travail-innovation-experimentation-formation-lp}{Site du groupe InEFLP}.
\sphinxhref{https://irem.univ-amu.fr/fr}{\sphinxincludegraphics[width=150\sphinxpxdimen]{{logo-irem-2013-blanc}.jpg}}\sphinxhref{https://irem.univ-amu.fr/fr}{\sphinxincludegraphics[width=150\sphinxpxdimen]{{amu}.png}}

\section{Table des matières du site}
\label{\detokenize{index:table-des-matieres-du-site}}

\subsection{GeoGebra et Moodle ?}
\label{\detokenize{projet-description::doc}}\label{\detokenize{projet-description:geogebra-et-moodle}}
Il existe un plugin pour \sphinxstyleemphasis{Moodle} %
\begin{footnote}[1]\sphinxAtStartFootnote
Voir le site officiel du plugin : \sphinxurl{https://moodle.org/plugins/mod\_geogebra}
%
\end{footnote} permettant de créer des activités \sphinxstyleemphasis{GeoGebra} et de sauvegarder son état. Il est alors possible :
\begin{itemize}
\item {} 
de sauvegarder la production d’un élève

\item {} 
d’évaluer manuellement ou \sphinxstylestrong{automatiquement} l’activité.

\end{itemize}


\subsubsection{GeoGebra}
\label{\detokenize{projet-description:geogebra}}
GeoGebra %
\begin{footnote}[2]\sphinxAtStartFootnote
Site GeoGebra : \sphinxurl{http://geogebra.org/}
%
\end{footnote} est un logiciel de géométrie dynamique permettant d’explorer, d’expérimenter mais aussi de créer de la ressource pour les élèves.

Nous utilisons \sphinxstyleemphasis{GeoGebra} pour créer des exerciseurs. Généralement, nos activités élèves sont construites de la façon suivante :
\begin{enumerate}
\item {} 
réaliser une tâche générée avec des \sphinxstylestrong{valeurs aléatoires}

\item {} 
saisir sa réponse et valider

\item {} 
si c’est incorrect, alors la réponse est affichée avec un corrigé

\item {} 
recommencer un certain nombre de fois les points 1 à 3

\item {} 
à la fin, obtenir un score sur son activité

\end{enumerate}

\begin{sphinxadmonition}{note}{\label{projet-description:index-0}À faire:}
Insérer une gif animée d’un exerciseur GGB
\end{sphinxadmonition}


\subsubsection{Moodle}
\label{\detokenize{projet-description:moodle}}
Moodle %
\begin{footnote}[3]\sphinxAtStartFootnote
Site francophone de Moodle : \sphinxurl{https://moodle.org/?lang=fr}
%
\end{footnote} est une application en ligne permettant de faire de la formation à distance. De nombreux MOOC actuels l’utilisent. Cet outil est disponible dans de nombreuses académie. Par exemple sur les académies de Nice et d’Aix-Marseille, Moodle est intégré à l’ENE Atrium %
\begin{footnote}[4]\sphinxAtStartFootnote
Présentation de l’intégration de Moodle (et autres) avec Atrium : \sphinxurl{https://www.atrium-paca.fr/web/assistance/acceder-a-moodle-chamilo-pronote-correlyce}
%
\end{footnote}.

Nous utilisons Moodle pour mettre les élèves en activités à l’aide d”\sphinxstylestrong{évaluations formatives}. En effet, la notation choisie permet à l’éléve de s’entraîner, de se former et d’être en réussite.

\begin{sphinxadmonition}{tip}{Astuce:}
Dans Moodle, nous utilisons la notation suivante :
\begin{itemize}
\item {} 
nombre maximum de tentatives : \sphinxstylestrong{illimité}

\item {} 
méthode d’évaluation : \sphinxstylestrong{Tentative la plus haute}

\end{itemize}

\noindent\sphinxincludegraphics[width=0.300\linewidth]{{fig-moodle}.png}
\end{sphinxadmonition}

Ainsi, l’élève fait autant de fois qu’il le désire l’actitivé proposée. Il n’est pas obligé d’aller à son terme si c’est une activité répétitive. L’élève apprend de ses erreurs car un corrigé l’accompagne à chaque tentative. Lorsqu’il le désire, l’élève recommence l’activité et tente alors d’avoir un score maximal.

Cette façon d’évaluer est pour nous très pertinente.
L’élève est en activité, il est motivé car il sait qu’il peut réussir.
Il gagne en autonomie car, grâce au corrigé, il cherche à comprendre et à ne pas reproduire ses erreurs.
En effet, chaque tentative est différente de la précédente car les fichiers sont conçus à partir de \sphinxstylestrong{valeurs aléatoires}.


\subsubsection{Le plugin GeoGebra pour Moodle}
\label{\detokenize{projet-description:le-plugin-geogebra-pour-moodle}}
Ce plugin offre un nouveau type d’activité dans Moodle : \sphinxstyleemphasis{GeoGebra}.


\paragraph{Utilisation de base}
\label{\detokenize{projet-description:utilisation-de-base}}
De base, il permet à l’enseignant de diffuser un fichier GeGebra (à envoyer dans l’onglet \sphinxstyleemphasis{Contenu}).
L’élève entre alors dans l’activité et peut :
\begin{itemize}
\item {} 
modifier le fichier

\item {} 
sauvegarder ses modifications pour y revenir plus tard

\item {} 
envoyer sa production à l’enseignant.

\end{itemize}

L’enseigant pourra alors consulter les productions de chaque élève (un peu long car à chaque fois le fichier doit s’ouvrir), ajouter commentaire et/ou note.


\paragraph{Utilisation avancée}
\label{\detokenize{projet-description:utilisation-avancee}}
Une utilisation avancée du plugin permet la \sphinxstylestrong{notation automatique}.

\begin{sphinxadmonition}{tip}{Astuce:}
Pour activer la notation automatique, il faut :
\begin{itemize}
\item {} 
dans Moodle
* cocher \sphinxstyleemphasis{Activité auto-évaluée} dans l’onglet \sphinxstyleemphasis{Note}
* définir la note maximale

\item {} 
dans GeoGebra
* créer une variable \sphinxstyleemphasis{grade} qui aura une valeur entre 0 et la note maximale
* incrémenter la variable grade en fonction de l’activité de l’élève

\end{itemize}

\noindent\sphinxincludegraphics[width=0.300\linewidth]{{fig-moodle2}.png}

\noindent{\hspace*{\fill}\sphinxincludegraphics[width=0.600\linewidth]{{fig-moodle3}.png}}
\end{sphinxadmonition}


\bigskip\hrule\bigskip



\subsection{Pourcentages}
\label{\detokenize{analyse alg_xe8bre - pourcentages:pourcentages}}\label{\detokenize{analyse alg_xe8bre - pourcentages::doc}}
Nous vous proposons 7 exercices sur les pourcentages.
Voici par exemple l’exercice 7 qui nous a servi de \sphinxstylestrong{synthèse}.

\begin{sphinxadmonition}{warning}{Avertissement:}
Attention, le bloc ci-dessous est dynamique et peut mal s’afficher…
\end{sphinxadmonition}




\subsubsection{Fichiers à télécharger}
\label{\detokenize{analyse alg_xe8bre - pourcentages:fichiers-a-telecharger}}

\begin{savenotes}\sphinxattablestart
\centering
\sphinxcapstartof{table}
\sphinxcaption{Pourcentages directs et indirects}\label{\detokenize{analyse alg_xe8bre - pourcentages:id1}}
\sphinxaftercaption
\begin{tabulary}{\linewidth}[t]{|T|T|}
\hline
\sphinxstyletheadfamily 
Fichier
&\sphinxstyletheadfamily 
Description
\\
\hline
\sphinxhref{\_static/exerciseur\_pourcentage1\_550\%C3\%97700.ggb}{pourcentage1.ggb}
&
pourcentage direct (calculer une quantité à partir du taux)
\\
\hline
\sphinxhref{\_static/exerciseur\_pourcentage2\_550\%C3\%97700.ggb}{pourcentage2.ggb}
&
calculer un taux (à partir des quantités initiales et finales)
\\
\hline
\sphinxhref{\_static/exerciseur\_pourcentage3\_550\%C3\%97700.ggb}{pourcentage3.ggb}
&
pourcentage indirect (calculer la quantité initiale à partir du taux)
\\
\hline
\end{tabulary}
\par
\sphinxattableend\end{savenotes}


\begin{savenotes}\sphinxattablestart
\centering
\sphinxcapstartof{table}
\sphinxcaption{Augmentations, diminutions de pourcentages}\label{\detokenize{analyse alg_xe8bre - pourcentages:id2}}
\sphinxaftercaption
\begin{tabular}[t]{|\X{1}{3}|\X{2}{3}|}
\hline

\sphinxhref{\_static/exerciseur\_pourcentage4\_550\%C3\%97700.ggb}{pourcentage4.ggb}
&
calculer une quantité après une augmentation/réduction
\\
\hline
\sphinxhref{\_static/exerciseur\_pourcentage5\_550\%C3\%97700.ggb}{pourcentage5.ggb}
&
calculer un taux d’augmentation/réduction
\\
\hline
\sphinxhref{\_static/exerciseur\_pourcentage6\_550\%C3\%97700.ggb}{pourcentage6.ggb}
&
calculer la quantité initiale connaissant la quantité finale et le taux d’augmentation/réduction
\\
\hline
\end{tabular}
\par
\sphinxattableend\end{savenotes}


\begin{savenotes}\sphinxattablestart
\centering
\sphinxcapstartof{table}
\sphinxcaption{Synthèse}\label{\detokenize{analyse alg_xe8bre - pourcentages:id3}}
\sphinxaftercaption
\begin{tabular}[t]{|\X{1}{3}|\X{2}{3}|}
\hline

\sphinxhref{\_static/exerciseur\_pourcentage7\_550\%C3\%97700\_totale.ggb}{pourcentage7.ggb}
&
la synthèse : un mélange de tous les cas précédents
\\
\hline
\end{tabular}
\par
\sphinxattableend\end{savenotes}


\subsubsection{Caractéristiques}
\label{\detokenize{analyse alg_xe8bre - pourcentages:caracteristiques}}\begin{itemize}
\item {} 
exercices sur 5 points (5 questions)

\item {} 
valeurs aléatoires (quantités, unités, questions, etc.)

\item {} 
notation automatique avec le plugin moodle : grâce à la variable \sphinxstyleemphasis{grade}

\end{itemize}


\subsection{Fonctions}
\label{\detokenize{analyse alg_xe8bre - fonctions:fonctions}}\label{\detokenize{analyse alg_xe8bre - fonctions::doc}}

\subsubsection{Fichiers à télécharger}
\label{\detokenize{analyse alg_xe8bre - fonctions:fichiers-a-telecharger}}

\begin{savenotes}\sphinxattablestart
\centering
\sphinxcapstartof{table}
\sphinxcaption{Calcul d’images}\label{\detokenize{analyse alg_xe8bre - fonctions:id1}}
\sphinxaftercaption
\begin{tabular}[t]{|\X{30}{100}|\X{70}{100}|}
\hline
\sphinxstyletheadfamily 
Fichiers
&\sphinxstyletheadfamily 
Description
\\
\hline
\sphinxhref{\_static/exerciseur\_AA-calcul-image-affine\%20-\%20460x537.ggb}{calcul-image-affine.ggb}
&
calculer l’image à partir de l’expression algébrique d’une fonction affine
\\
\hline
\sphinxhref{\_static/exerciseur\_AA-calcul-image-polynome\%20-\%20522x419.ggb}{calcul-image-polynôme.ggb}
&
calculer l’image à partir de l’éxpression algébrique d’une fonction rationnelle
\\
\hline
\end{tabular}
\par
\sphinxattableend\end{savenotes}


\begin{savenotes}\sphinxattablestart
\centering
\sphinxcapstartof{table}
\sphinxcaption{Sens de variation}\label{\detokenize{analyse alg_xe8bre - fonctions:id2}}
\sphinxaftercaption
\begin{tabular}[t]{|\X{30}{100}|\X{70}{100}|}
\hline
\sphinxstyletheadfamily 
Fichiers
&\sphinxstyletheadfamily 
Description
\\
\hline
\sphinxhref{\_static/exerciseur\_sens\%20de\%20variation\%20fonction\%20affine\_\%20graphique.ggb}{sens de variation fonction affine graphique.ggb}
&
déterminer le sens de variation d’une fonction affine à partir de sa représentation graphique
\\
\hline
\sphinxhref{\_static/exerciseur\_sens\%20de\%20variation\%20fonction\%20affine\_.ggb}{sens de variation fonction affine.ggb}
&
déterminer le sens de variation d’une fonction affine à partir de l’expression algébrique
\\
\hline
\end{tabular}
\par
\sphinxattableend\end{savenotes}


\subsubsection{Caractéristiques}
\label{\detokenize{analyse alg_xe8bre - fonctions:caracteristiques}}\begin{itemize}
\item {} 
exercices sur 5 points (5 questions)

\item {} 
valeurs aléatoires (quantités, unités, questions, etc.)

\item {} 
notation automatique avec le plugin moodle : grâce à la variable \sphinxstyleemphasis{grade}

\end{itemize}


\subsection{Vecteurs}
\label{\detokenize{geom-vecteur::doc}}\label{\detokenize{geom-vecteur:vecteurs}}
\noindent{\hspace*{\fill}\sphinxincludegraphics[width=0.650\linewidth]{{fig-vecteur}.png}\hspace*{\fill}}


\subsubsection{Fichiers à télécharger}
\label{\detokenize{geom-vecteur:fichiers-a-telecharger}}

\begin{savenotes}\sphinxattablestart
\centering
\sphinxcapstartof{table}
\sphinxcaption{Coordonnées et normes dans le plan}\label{\detokenize{geom-vecteur:id1}}
\sphinxaftercaption
\begin{tabular}[t]{|\X{2}{4}|\X{2}{4}|}
\hline
\sphinxstyletheadfamily 
Fichier
&\sphinxstyletheadfamily 
Description
\\
\hline
\sphinxhref{\_static/exerciseur\_coordonnées\%20vecteur\%20plan.ggb}{coordonnées vecteur plan.ggb}
&
déterminer les coordonnées d’un vecteur plan
\\
\hline
\sphinxhref{\_static/exerciseur\_norme\%20vecteur\%20plan.ggb}{norme vecteur plan.ggb}
&
calculer la norme d’un vecteur dans le plan (à partir des coordonnées)
\\
\hline
\end{tabular}
\par
\sphinxattableend\end{savenotes}


\subsubsection{Caractéristiques}
\label{\detokenize{geom-vecteur:caracteristiques}}\begin{itemize}
\item {} 
exercices sur 5 points (5 questions)

\item {} 
valeurs aléatoires (quantités, unités, questions, etc.)

\item {} 
notation automatique avec le plugin moodle : grâce à la variable \sphinxstyleemphasis{grade}

\item {} 
correction personnalisée

\end{itemize}


\subsection{Ressources en lignes}
\label{\detokenize{ressources:ressources-en-lignes}}\label{\detokenize{ressources::doc}}
Nous remercions chaleureusement Joël Gauvain, fondateur du site \sphinxstylestrong{Mathématiques à Valin}, sans qui nous n’aurions pas sû utiliser cet outil formidable : GeoGebra + Moodle !

Voici un lien vers ses explications et ses ressources : énorme !

\sphinxhref{http://lycee-valin.fr/maths/exercices\_en\_ligne/moodle.html}{Mathématiques à Valin, Moodle et GeoGebra}



\renewcommand{\indexname}{Index}
\printindex
\end{document}